%%
%% Abstract
%%
%%%%%%%%%%%%%%%%%%%%%%%%%%%%%%%%%%%%%%%%%%%%%%%%%%%%%%%%%%%%%%%%%%%%%


\section*{Kurzfassung}
Mit der enormen Zunahme von Information in unterschiedlichen Quellen wie Sensoren (RFID) oder 
Nachrichtenquellen (RFD newsfeeds) wird es schwieriger Informationen beständig abzufragen. Um die
Frage zu klären, welcher Rechner am häufigsten in einem Netzwerk frequentiert wird, werden unterstützende 
Systeme notwendig. An dieser Stelle helfen Methoden aus dem Bereich des Complex Event Processing (CEP).
Im Spezialbereich Stream Processing von CEP wurden Streaming Frameworks entwickelt, 
um die Arbeit in der Datenflussverarbeitung zu unterstützen und damit komplexe Abfragen auf einer höheren
Schicht zu vereinfachen. In dieser Master Thesis geht es um einen Vergleich zwischen den Streaming Frameworks:
Apache Storm, Apache Kafka, Apache Flume und Apache S4. In der Thesis werden Prototypen entwickelt, um
Messwerte zu erfassen und zu vergleichen.



%% eof
