\section{Installationsanleitung Aurora/Borealis}
\label{sec:aurborinstall}


In dieser Anleitung wird die Installation von Aurora Borealis in kleinen Unterkapiteln vorgestellt. Diese Anleitung setzt ein Vorwissen in der Verwendung und Administration von Linux voraus. Zudem wird Erfahrung von Erzeugen von Anwendungen aus Quelltext benötigt. Zum Beispiel können Konflikte auftreten, wenn neue Versionen von Bibliotheken benötigt werden. Dabei müssen die Abhängigkeiten beachtet und abhängige Konflikte aufgelöst werden. Bevor das Erstellen der Anwendung beginnt werden zuerst die Voraussetzungen bestimmt und erläutert. Anschließend wird mit Quelltextfragementen und Kommandozeilenausschnitten schrittweise die Eingabe und Ausgabe gezeigt.


\subsection{Voraussetzungen am Betriebssystem}

Die Installation von Aurora Borealis benötigt ein auf linuxbasiertes Betriebsystem. Auf dem Betriebssystem Microsoft Windows wurde eine Erstellung des Quelltextes von Aurora Borealis bisher nicht durchgeführt. Zum Zeitpunkt der Erstellung dieser Anleitung wird versucht ein Ist-Zustand der Anwendung aufzunehmen. Für das Verteilte System Aurora Borealis wird die Linuxdistribution Debian benutzt.


\subsection{Voraussetzung Erstellsystem}

Einige Pakete bzw. Bibliotheken werden für das Erstellen von Aurora Borealis benötigt. Als Paketverwaltung wird unter Debian \textit{apt} benutzt. Mit dem Befehl \textit{apt-get} können Pakete dem Betriebssystem aus dem Standard Debian Paket-Repository hinzugefügt werden. Folgende Liste zeigt benötigte Pakete für das Erstellen von Aurora Borealis:

\begin{itemize}
	\item build-essentials (gcc, g++, configure, make)
	\item ccache
	\item antlr
	\item libxerces-c3.1 (Xerces-c: Used by Borealis to parse XML)
	\item libtool
	\item autoconf
	\item automake
	\item libdb5.1 (Berkeley-Db)
	\item glpk (GNU Linear Programming Kit)
	\item gsl (GNU Scientific Library - collection of routines for numerical analysis: used for predictive queries)
	\item opencv (open source computer vision: used for array processing)
	\item doxygen (serves to generate documentation)
	\item openjdk-7-jdk (java 7)
\end{itemize}

Da die letzte Version von Aurora Borealis aus dem Jahr 2008 ist, gibt es beim Erstellen mit neueren Versionen von \textit{gcc} und \textit{g++} Fehler. 
Damit die neue Version von \textit{gcc} benutzt werden kann muss der Quelltext der Version aus 2008 angepasst werden. Eine erste Anpassung wurde versucht durchzuführen. Zum Beispiel sind bestimmte Standardmethoden direkt angegeben worden. Trotzdem wurden weiterführende Fehler gefunden. Als Fehler wird \textit{missing \#include} gemeldet. Im \textit{borealis} Verzeichnis sind 239 \textit{Makefiles} vorhanden. Alle \textit{Makefiles} und die darin verbundenen Quelltext-Dateien müssen auf die neue Version geprüft werden. Eine Stabile Version mit den neuen Anpassungen kann nur durch effektive Testläufe gewährleistet werden. Um den Quelltext von Aurora Borealis nicht anzupassen kommt die ältere Version 4.0 von Debian zum Einsatz. In diesem Fall muss die Liste von benötigten Paketen angepasst werden:

\begin{itemize}
	\item build-essentials (gcc, g++, configure, make)
	\item ccache
	\item antlr
	\item libxerces27 (Xerces-c: Used by Borealis to parse XML)
	\item libtool
	\item autoconf
	\item automake
	\item libdb (Berkeley-Db)
	\item glpk (GNU Linear Programming Kit)
	\item gsl (GNU Scientific Library - collection of routines for numerical analysis: used for predictive queries)
	\item opencv (open source computer vision: used for array processing)
	\item doxygen (serves to generate documentation)
	\item sun-java5-jdk (Java 1.5 von SUN)
	\item libexpat1-dev
	\item ibreadline5-dev
\end{itemize}


\subsection{Quelltext von Aurora Borealis herunterladen}

Die Datei liegt nicht in einer öffentlichen Versionsverwaltung, sondern kann als Archive von der Brown University unter folgendem Link heruntergeladen werden:
\url{http://www.cs.brown.edu/research/borealis/public/download/borealis_summer_2008.tar.gz} \\
Alternativ liegt der Quelltext und das Installationsmedium Debian 4.0 als \gls{glo:iso} 9660 Abbild im Ordner \textit{anhangSoftwareZusatz}. Nachdem die Anwendung im Verzeichnis \textit{/opt} liegt kann sie im gleichen Verzeichnis entpackt werden. Im Verzeichnis liegen anschließend zwei Unterordner \textit{borealis} und \textit{nmstl}.


\subsection{Kommandozeile Umgebungsvariablen festlegen}

Unter Debian wird für die Kommandozeile die Shell \textit{Bash} eingesetzt. Wenn eine Shell eröffnet wird, wird die Datei \textit{.bashrc} im Benutzerverzeichnis aufgerufen. Darin werden Benutzerabhängige Konfigurationen abgelegt. Die Umgebungsvariablen für Aurora Borealis werden im folgenden Abschnitt gezeigt:
\begin{verbatim}
alias debug='export LOG_LEVEL=2'
alias debug0='export LOG_LEVEL=0'
export PATH=${PATH}:/opt/nmstl/bin:/${HOME}/bin
export CLASSPATH='.:/usr/share/java/antlr.jar:$CLASSPATH'
export JAVA_HOME=$(readlink -f /usr/bin/javac | sed "s:/bin/javac::")
export PATH=${JAVA_HOME}/bin:${PATH}
export CXX='ccache g++'
export CVS_SANDBOX='/opt'
export INSTALL_BDB='/usr'
export INSTALL_GLPK='/usr'
export INSTALL_GSL='/usr'
export INSTALL_ANTLR='/usr'
export INSTALL_XERCESC='/usr'
export INSTALL_NMSTL='/usr/local'
export LD_LIBRARY_PATH='/usr/lib'
export ANTLR_JAR_FILE='/usr/share/java/antlr.jar'

mkdir -p bin
alias bbb='/opt/borealis/utility/unix/build.borealis.sh'
alias bbbt='/opt/borealis/utility/unix/build.borealis.sh -tool.head
 -tool.marshal'
alias retool='/bin/cp -f ${CVS_SANDBOX}/borealis/tool/head/BigGiantHead
 ${HOME}/bin; /bin/cp -f ${CVS_SANDBOX}/borealis/tool/marshal/marshal
 ${HOME}/bin'
\end{verbatim}


\subsection{Notwendige Quelltext Anpassung}

Beim Erzeugen wird unter anderen Paketen auch \gls{glo:antlr} benutzt. Während der Erstellens findet ein Fehler auf. Bei der Konfiguration für das \textit{Makefile} wurde ein Pfad fest einprogrammiert. Dieser feste Pfad wird nun durch einen Variable in die Umgebungsvariable \textit{ANTLR\_JAR\_FILE} ausgelagert. Dazu muss in der Datei \textit{/opt/borealis/src/configure.ac} der Inhalt an der Stelle ANTLR\_JAR\_FILE= mit \textit{\$ANTLR\_JAR\_FILE} nach dem Gleichzeichen ausgetauscht werden.


\subsection{Erzeugen von NMSTL}

Borealis benutzt eine angepasste Version von \gls{glo:nmstl}. Im Verzeichnis \textit{/opt/nmstl} werden nun folgende Befehle nacheinander ausgeführt:

\begin{verbatim}
autoconf
./configure
make
make install
\end{verbatim}


\subsection{Erzeugen von Borealis}

In das Verzeichnis \textit{/opt/borealis} zurückspringen und folgende Befehle nacheinander ausführen:

\begin{verbatim}
bbb
bbbt
retool
make install
\end{verbatim}


\subsection{Zusätzliche Informationen}
Der Support für Aurora Borealis ist seit 2008 eingestellt. Weitere Informationen stehen unter folgendem Link:
\url{http://cs.brown.edu/research/borealis/public/install/install.borealis.html}
